\documentclass[]{article}
\usepackage{lmodern}
\usepackage{amssymb,amsmath}
\usepackage{ifxetex,ifluatex}
\usepackage{fixltx2e} % provides \textsubscript
\ifnum 0\ifxetex 1\fi\ifluatex 1\fi=0 % if pdftex
  \usepackage[T1]{fontenc}
  \usepackage[utf8]{inputenc}
\else % if luatex or xelatex
  \ifxetex
    \usepackage{mathspec}
  \else
    \usepackage{fontspec}
  \fi
  \defaultfontfeatures{Ligatures=TeX,Scale=MatchLowercase}
\fi
% use upquote if available, for straight quotes in verbatim environments
\IfFileExists{upquote.sty}{\usepackage{upquote}}{}
% use microtype if available
\IfFileExists{microtype.sty}{%
\usepackage{microtype}
\UseMicrotypeSet[protrusion]{basicmath} % disable protrusion for tt fonts
}{}
\usepackage[margin=1in]{geometry}
\usepackage{hyperref}
\hypersetup{unicode=true,
            pdftitle={Evolutionary game theory: A modified Ultimatum game model with algae as players},
            pdfborder={0 0 0},
            breaklinks=true}
\urlstyle{same}  % don't use monospace font for urls
\usepackage{graphicx,grffile}
\makeatletter
\def\maxwidth{\ifdim\Gin@nat@width>\linewidth\linewidth\else\Gin@nat@width\fi}
\def\maxheight{\ifdim\Gin@nat@height>\textheight\textheight\else\Gin@nat@height\fi}
\makeatother
% Scale images if necessary, so that they will not overflow the page
% margins by default, and it is still possible to overwrite the defaults
% using explicit options in \includegraphics[width, height, ...]{}
\setkeys{Gin}{width=\maxwidth,height=\maxheight,keepaspectratio}
\IfFileExists{parskip.sty}{%
\usepackage{parskip}
}{% else
\setlength{\parindent}{0pt}
\setlength{\parskip}{6pt plus 2pt minus 1pt}
}
\setlength{\emergencystretch}{3em}  % prevent overfull lines
\providecommand{\tightlist}{%
  \setlength{\itemsep}{0pt}\setlength{\parskip}{0pt}}
\setcounter{secnumdepth}{5}
% Redefines (sub)paragraphs to behave more like sections
\ifx\paragraph\undefined\else
\let\oldparagraph\paragraph
\renewcommand{\paragraph}[1]{\oldparagraph{#1}\mbox{}}
\fi
\ifx\subparagraph\undefined\else
\let\oldsubparagraph\subparagraph
\renewcommand{\subparagraph}[1]{\oldsubparagraph{#1}\mbox{}}
\fi

%%% Use protect on footnotes to avoid problems with footnotes in titles
\let\rmarkdownfootnote\footnote%
\def\footnote{\protect\rmarkdownfootnote}

%%% Change title format to be more compact
\usepackage{titling}

% Create subtitle command for use in maketitle
\newcommand{\subtitle}[1]{
  \posttitle{
    \begin{center}\large#1\end{center}
    }
}

\setlength{\droptitle}{-2em}
  \title{Evolutionary game theory: A modified Ultimatum game model with algae as
players}
  \pretitle{\vspace{\droptitle}\centering\huge}
  \posttitle{\par}
  \author{}
  \preauthor{}\postauthor{}
  \predate{\centering\large\emph}
  \postdate{\par}
  \date{2018}

\usepackage{enumerate}

\begin{document}
\maketitle

We have created a mathematical model, the Enhanced Ultimatum Game, and
have written a script in python of the model. The simulation may be used
to predict how an algae population will evolve and, in turn, how much
chlorophyll a we may expect.

\section{What is the Ultimatum Game
(UG)?}\label{what-is-the-ultimatum-game-ug}

The Ultimatum Game (UG) is game theoretical model of fairness and is a
focal point for studies in the evolution of social behaviour. In the UG,
two players must decide on the division of resources. In contrast to the
predictions of traditional game theory, participants appear to make
irrational decisions; participants tend to divide the resources equally
and would rather suffer a loss than to accept an unfair division. Here,
we investigate a new UG theoretical model, the Enhanced Ultimatum Game
(EUG) where we introduce a cost associated with a demand.

This model and other agent-based models are not limited to predicting
human behaviour. Past studies have included such models to simulate
behaviour of animals (Karsaia, Montanoa and Schmicklb,2016), plants
(King, D. 1990), molecules (Bohl et al. (2014)) and more (Oswald \&
Schmikl, 2017; Zahadat and Schmickl (2016)). In these instances, what is
of interest is the interaction of the members within a population and
the changes that occur. It is my understanding that, if a particular
population evolves via interactions of the members within its group,
then the UG model may be applied.

For the purpose of a simple description, let's consider how the UG game
would be applied to people. The Ultimatum Game (UG) involves two
players, the proposer and the responder, deciding on how to divide a
resource between them. The proposer first offers an amount to the
responder. If the responder accepts the offer then the resource is
divided according to the proposal. If the offer is rejected then both
players walk away with nothing.

In the Enhanced Ultimatum Game (EUG), the responder takes on an
additional role as the demander. In this version, the responder first
makes a demand for how to divide the resources and the proposer will
then make an offer. The responder will then accept, with a cost, or
reject.

In the case of an algae population, there is limited resource available
within an environment which is to be divided among the algae. There
exists competition for resources and certain types of algae will be more
likely to succeed given particular enviroments. The UG model provides a
method for classifying types of algae and how interaction may evolve the
population. For instance, let's consider surface area of an alga X and
suppose one of the resources to be divided is the sun. Suppose X has the
largest surface area compared to other members in the population. Since
surface area will play a large role in dictating the amount of nutrients
algae may absorb from the sun, X is at advantage in this regard. We may
interpret X as having a high Demand. Next suppose X is shaded from the
sun for a finite amount of time. If X has some other means of obtaining
necessary nutrients then it will continue to thrive. We may interpret X
as having a low Min accept. X then has the best strategy and is more
likely to survive and dominate a population.

The EUG incorporates a demand and cost into the game and is played
between two participants as follows:

\begin{itemize}
    \item There are $n$ resources to be divided 
    \item The responder demands an amount $d$ such that $0<d<n$
    \item The proposer then makes an offer $p$
    \item The responder has a minimum $M$ that they are willing to accept such that $M \leq d$
    \item There is a cost to the responder when $d \neq M$. The cost is $c(d-M)$ where $c>0$
    \item If $p \leq M$ then the resources are divided accordingly. If $P>M$ then both receive zero resources
\end{itemize}

\section{How has the UG model been modified for this project? How is
this model unique from
literature?}\label{how-has-the-ug-model-been-modified-for-this-project-how-is-this-model-unique-from-literature}

The EUG model is a modified version of the model found within the
literature, the UG model. Also, for this project, we are interested in
the amount of Chlorophyll a within a lake and so, for the simulation,
algae will act as the players in the EUG model. Also, several factors
influence the evolution of chlorophyll in an algae population (ie.
nutrients, sunlight, total phosphate etc). For instance, Total Phosphate
(TP) found within a lake is thought to influence the quantity of a type
of algae and sun exposure is associated with specific chlorophyll types.
Therefore, for the purposes of this project, one may view TP and
sunlight exposure as the resource to be divided in the EUG model. This
will allow us to make predictions about the group research question:
Whether the quantity of Chlorophyll a changes over time in a lake, given
a particular environment.

\section{Brief description of
simulation}\label{brief-description-of-simulation}

The simulated data will be generated according to a variation on
Agent-based modelling (ABM) that is commonly used in ecology, referred
to as invidivual-based modelling (IBM) . In particular, we shall use the
parameter of resource division to simulate the evolution of an algae
population in a lake according to individual behavior. The data obtained
will be used to make predictions about how an enviroment may change with
respect to Chlorophyll a, given certain environmental conditions and an
initial measure of Chlorophyll a.

\section{In what program was the model
coded?}\label{in-what-program-was-the-model-coded}

I created my main code in python.

\section{What are the parameters?}\label{what-are-the-parameters}

There are several possibilities (perhaps infinite) for parameter
settings and so in the initial stages I was working out exactly what I
wanted to analyse. Below is a list of the more interesting ways the
simulation was altered:

\begin{itemize}
    \item Created a data set from various distributions (ie. poisson distribution and varied lambda)
    \item Varied the cost parameter (ie. instead of n(D-MA) for some n>0, I tried n(D-2P+MA) for some n>0 etc).
    \item Introduced a resource probability parameter such that the resource available to any one alga would vary according to a normal distribution.
    \item Created an algae-type parameter (based on chlorophyll a vs. b) that was associated with proposal values.  
    \item there ended up being many versions of this code! 
\end{itemize}

The main and more interesting graphs are stored within the folder titled
`Plots with varying parameters'. I may share more, if they are of
interest.

For the graph displayed on the poster, the following parameters were
used:

\begin{itemize}
  \item resource = 20           (Amount of resource to be divided)
  \item cost = 2              (Cost for demanding more than MA)
  \item runs = 10                (Number of interactions)
  \item popsize = 500          (population size/Number of algae)
  \item generations = 500       (Number of generations)
  \item tsize = 4              (Tournament size)
  \item pmr=0.05            (Point mutation rate for proposal)
  \item mdr=0.5             (Point mutation rate for demand and MA)
  \item epoch=5
\end{itemize}

\section{Algorithm/psuedocode}\label{algorithmpsuedocode}

Below is an outline/description of the python code:

\begin{enumerate}[\{1\}]
\item Import necessary packages (numpy and matplotlib.pyplot)
\item Define variables (resource, cost, runs, popsize, generations, tsize, pmr, mdr, epoch)
\item Create main loop. 
    \begin{enumerate}[i)]
        \item Create characteristics of population: create an array for each alga within the population. Within each array describes the structure of the alga. The array contains 23 cells [3,4,2,6,2,14,15,13,2,6,7,8,6,4,3,3,7,8,9,9,7,18,0].
The first 20 cells correspond to proposal values that the alga will make, dependent on the demands. For example, if a second agla demands 3, then the first alga will propose the amount found in the 4th cell of the array. If the second alga demands 4, then the first alga will propose the amount found in the 5th cell of its array, and so on. The 21st cell is the alga's minaccept value, the 22nd is it's demand value, and the 23rd slot is it's fitness score. Suppose there is a population of n algae then n 23 celled arrays are created.   
        \item Population interacts: Have each member of the population interact with all other members of the population twice, once as the demander and again as the proposer. Calculate and append fitness scores. Continue for g generations. 
        \item Point mutation: Pull out a few members of the population, choose the two members with the highest fitness scores and have them produce two "offspring" that will replace two existing algae that have the lowest fitness scores.
        \item Stats: calculate average, minimum and maximum fitness scores across epochs
        \item Plot: Create graph of maximum, minimum, and average fitness scores vs epoch
            \end{enumerate}
\end{enumerate}

\section{How were the parameters
varied?}\label{how-were-the-parameters-varied}

For the graph displayed on the poster, all values found in the section
of this paper titled ``What are the parameters'' were varied. However, I
eventually fixed most values, as mentioned, except for the ``runs'',
``generations'' and ``epochs'' variables. I varied the parameters within
the range that the memory of my computer would allow (approx. 0 -
100,000 units).

\section{Any interesting findings?}\label{any-interesting-findings}

It has proven difficult (or thought provoking?) to interpret a couple of
the results. I found what presents as an expected Nash equilibrium such
that a population will remain stable if members of the population accept
anything while demanding as much as possible. This is seen in the plot
(albeit labelled incorrectly as Absorbance) where the minaccept(green)
is reducing over time but the demand (blue) is slightly higher. It can
be seen that this trend continues when increasing the number of
generations. However, if the parameters are altered, the findings match
closer to what we would expect to be found with people and the division
of resource. In this case, the strategy that appears to dominate the
population is one that is offering a 50/50 split of the resource.This
can be seen in the Proposalvsdemand plot where the amount proposed is
half the resource.

The main finding: The simulation predicts that the algae population will
tend towards a population with a higher ``fitness'' score. Given how
``fitness'' was defined for this model, we may expect the quantiy of
chlorophyll a to grow substantially, as higher fitness implies more
algae which implies more chlorophyll a. It's also interesing to note
that the ``fitness'' score grows but appears to stabilize at a fixed
value, suggesting an equilibrium is met. It's possible that the lake
reaches a carrying capacity which could explain the fixed value. Likely
what we're seeing is a particular type of algae (or some symbiotic
relationship or a characteristic shared) dominating the population.

\section{References}\label{references}

(For a complete list of references, see the ``Literature'' folder)

Carlson, R.E. and J. Simpson. 1996. A Coordinator's Guide to Volunteer
Lake Monitoring Methods. North American Lake Management Society. 96 pp.

\hypertarget{refs}{}
\hypertarget{ref-Bohl_2014}{}
Bohl, Katrin, Sabine Hummert, Sarah Werner, David Basanta, Andreas
Deutsch, Stefan Schuster, and Anja Schroeter. 2014. ``Evolutionary Game
Theory: Molecules as Players.'' \emph{Mol. BioSyst.} 10 (12). Royal
Society of Chemistry (RSC): 3066--74.
doi:\href{https://doi.org/10.1039/c3mb70601j}{10.1039/c3mb70601j}.

\hypertarget{ref-Zahadat_2016}{}
Zahadat, Payam, and Thomas Schmickl. 2016. ``Division of Labor in a
Swarm of Autonomous Underwater Robots by Improved Partitioning Social
Inhibition.'' \emph{Adaptive Behavior} 24 (2). SAGE Publications:
87--101.
doi:\href{https://doi.org/10.1177/1059712316633028}{10.1177/1059712316633028}.


\end{document}
