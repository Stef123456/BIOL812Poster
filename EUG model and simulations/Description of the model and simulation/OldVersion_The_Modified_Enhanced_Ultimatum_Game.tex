\documentclass[]{article}
\usepackage{lmodern}
\usepackage{amssymb,amsmath}
\usepackage{ifxetex,ifluatex}
\usepackage{fixltx2e} % provides \textsubscript
\ifnum 0\ifxetex 1\fi\ifluatex 1\fi=0 % if pdftex
  \usepackage[T1]{fontenc}
  \usepackage[utf8]{inputenc}
\else % if luatex or xelatex
  \ifxetex
    \usepackage{mathspec}
  \else
    \usepackage{fontspec}
  \fi
  \defaultfontfeatures{Ligatures=TeX,Scale=MatchLowercase}
\fi
% use upquote if available, for straight quotes in verbatim environments
\IfFileExists{upquote.sty}{\usepackage{upquote}}{}
% use microtype if available
\IfFileExists{microtype.sty}{%
\usepackage{microtype}
\UseMicrotypeSet[protrusion]{basicmath} % disable protrusion for tt fonts
}{}
\usepackage[margin=1in]{geometry}
\usepackage{hyperref}
\hypersetup{unicode=true,
            pdftitle={Evolutionary game theory: A modified Ultimatum game model with algae as players},
            pdfauthor={Stefanie Knebel},
            pdfborder={0 0 0},
            breaklinks=true}
\urlstyle{same}  % don't use monospace font for urls
\usepackage{graphicx,grffile}
\makeatletter
\def\maxwidth{\ifdim\Gin@nat@width>\linewidth\linewidth\else\Gin@nat@width\fi}
\def\maxheight{\ifdim\Gin@nat@height>\textheight\textheight\else\Gin@nat@height\fi}
\makeatother
% Scale images if necessary, so that they will not overflow the page
% margins by default, and it is still possible to overwrite the defaults
% using explicit options in \includegraphics[width, height, ...]{}
\setkeys{Gin}{width=\maxwidth,height=\maxheight,keepaspectratio}
\IfFileExists{parskip.sty}{%
\usepackage{parskip}
}{% else
\setlength{\parindent}{0pt}
\setlength{\parskip}{6pt plus 2pt minus 1pt}
}
\setlength{\emergencystretch}{3em}  % prevent overfull lines
\providecommand{\tightlist}{%
  \setlength{\itemsep}{0pt}\setlength{\parskip}{0pt}}
\setcounter{secnumdepth}{5}
% Redefines (sub)paragraphs to behave more like sections
\ifx\paragraph\undefined\else
\let\oldparagraph\paragraph
\renewcommand{\paragraph}[1]{\oldparagraph{#1}\mbox{}}
\fi
\ifx\subparagraph\undefined\else
\let\oldsubparagraph\subparagraph
\renewcommand{\subparagraph}[1]{\oldsubparagraph{#1}\mbox{}}
\fi

%%% Use protect on footnotes to avoid problems with footnotes in titles
\let\rmarkdownfootnote\footnote%
\def\footnote{\protect\rmarkdownfootnote}

%%% Change title format to be more compact
\usepackage{titling}

% Create subtitle command for use in maketitle
\newcommand{\subtitle}[1]{
  \posttitle{
    \begin{center}\large#1\end{center}
    }
}

\setlength{\droptitle}{-2em}
  \title{Evolutionary game theory: A modified Ultimatum game model with algae as
players}
  \pretitle{\vspace{\droptitle}\centering\huge}
  \posttitle{\par}
  \author{Stefanie Knebel}
  \preauthor{\centering\large\emph}
  \postauthor{\par}
  \predate{\centering\large\emph}
  \postdate{\par}
  \date{2018}


\begin{document}
\maketitle

\section{What is the Ultimatum Game
(UG)?}\label{what-is-the-ultimatum-game-ug}

The Ultimatum Game (UG) is game theoretical model of fairness and is a
focal point for studies in the evolution of social behaviour. In the UG,
two players must decide on the division of resources. In contrast to the
predictions of traditional game theory, participants appear to make
irrational decisions; participants tend to divide the resources equally
and would rather suffer a loss than to accept an unfair division. Here,
we investigate a new UG theoretical model for the evolution of fairness,
the Enhanced Ultimatum Game (EUG) where we introduce a cost associated
with a demand.

The Ultimatum Game (UG) involves two players, the proposer and the
responder, deciding on how to divide a resource between them. The
proposer first offers an amount to the responder. If the responder
accepts the offer then the resource is divided according to the
proposal. If the offer is rejected then both players walk away with
nothing.

In the Enhanced Ultimatum Game (EUG), the responder takes on an
additional role as the demander. In this version, the responder first
makes a demand for how to divide the resources and the proposer will
then make an offer. The responder will then accept, with a cost, or
reject.

The EUG models fairness as both players have the opportunity to be fair,
split the resource evenly, or unfair, attempt to take a larger portion.
Rejection of an unfair offer is interpreted as a form of punishment for
deviating from fairness. A truly rational player would seek to maximize
their payoff and should accept any offer which results in a positive
payoff, fair or unfair, and to demand as much as possible. However,
several models have shown that participants commonly make fair offers
and reject unfair offers, leading to fairness as a potential
evolutionary outcome.

This model and other agent-based models have been used to simulate
behaviour of animals, plants, molecules and more.

The EUG incorporates a demand and cost into the game and is played
between two participants as follows:

\begin{itemize}
    \item There are $n$ dollars to be divided 
    \item The responder demands an amount $d$ such that $0<d<n$
    \item The proposer then makes an offer $p$
    \item The responder has a minimum, $M$, that they are willing to accept such that $M \leq d$
    \item There is a cost to not giving a $d = M$. cost $= c(d-M)$, $c>0$.
    \item If $p \leq M$ then the dollars are divided accordingly. If $P>M$ then both receive zero dollars
\end{itemize}

Michelutti et al. (2009)

\section{How has the UG model been modified for this project? How is
this model unique from
literature?}\label{how-has-the-ug-model-been-modified-for-this-project-how-is-this-model-unique-from-literature}

The EUG model is a modified version of the model found within the
literature, the UG model. Also, for this project, we are interested in
the amount of Chlorophyll a within a lake and so, for the simulation,
algae will act as the players in the EUG model. Also, several factors
influence the evolution of chlorophyll in an algae population (ie.
nutrients, sunlight, total phosphate etc). For instance, Total Phosphate
(TP) found within a lake is thought to influence the quantity of a type
of algae and sun exposure is associated with specific chlorophyll types.
Therefore, for the purposes of this project, one may view TP and
sunlight exposure as the resource to be divided in the EUG model. This
is following an approach similar to the papers titles. This will allow
us to make predictions about the group research question: Whether
Chlorophyll a changes over time in a lake, given a particular
environment.

\section{Brief description of
simulation}\label{brief-description-of-simulation}

The simulated data will be generated according to a variation on
Agent-based modelling (ABM) that is commonly used in ecology, referred
to as invidivual-based modelling (IBM) . In particular, we shall use the
parameter of absorbance to simulate the evolution of an algae population
in a lake according to individual behavior. The data obtained will be
used to make predictions about how the enviroment will change according
to varying levels of absorbance

\section{In what program was the model
coded?}\label{in-what-program-was-the-model-coded}

I created my main code in python.

\section{What are the parameters?}\label{what-are-the-parameters}

There are several possibilities (perhaps infinite) for parameter
settings and so in the initial stages I was working out exactly what I
wanted to analyse. Below is a list of the more interesting ways the
simulation was altered:

\begin{itemize}
    \item created a data set from various distributions (ie. poisson distribution)
\end{itemize}

Only a few graphs are stored within the folder titled `Plots with
varying parameters' as it is not meant to be the point of
interest.\textbackslash{} For the graph displayed on the poster, the
following parameters were used:

\section{Algorithm/psuedocode}\label{algorithmpsuedocode}

Below is a outline/description of the python code:

\begin{enumerate}[\{1\}]
\item Import necessary packages (numpy and matplotlib.pyplot)
\item Define variables (resource, cost, runs, popsize, generations, tsize, pmr, mdr, epoch)
\item Create main loop. 
    \begin{enumerate}[i)]
        \item Create characteristics of population: create an array for each algae within the population. Within each array describes the structure of the algae. The array contains 23 cells [3,4,2,6,2,14,15,13,2,6,7,8,6,4,3,3,7,8,9,9,7,18,0].
The first 20 cells correspond to proposal values that the alga will make, dependent on the demands. For example, if a second agla demands 3, then the first alga will propose the amount found in the 4th cell of the array. If the second alga demands 4, then the first alga will propose the amount found in the 5th cell of its array, and so on. The 21st cell is the alga's minaccept value, the 22nd is it's demand value, and the 23rd slot is it's fitness score. Suppose there is a population of n algae then n 23 celled arrays are created.   
        \item Population interacts: Have each member of the population interact with all other members of the population twice, once as the demander and again as the proposer. Calculate and append fitness scores. Continue for g generations. 
        \item Point mutation: Pull out a few members of the population, choose the two members with the highest fitness scores and have them produce two "offspring" that will replace two existing algae that have the lowest fitness scores.
        \item Stats: calculate average, minimum and maximum fitness scores across epochs
        \item Plot: Create graph of maximum, minimum, and average fitness scores vs epoch
            \end{enumerate}
\end{enumerate}

\section{How were the parameters
varied?}\label{how-were-the-parameters-varied}

For the graph displayed on the poster, was run for values of

\section{Any interesting findings?}\label{any-interesting-findings}

It was found

\section{References}\label{references}

Carlson, R.E. and J. Simpson. 1996. A Coordinator's Guide to Volunteer
Lake Monitoring Methods. North American Lake Management Society. 96
pp.\textbackslash{} \textbackslash{} For a complete list of references,
see the ``Literature'' folder.

\hypertarget{refs}{}
\hypertarget{ref-Michelutti_2009}{}
Michelutti, Neal, Jules M. Blais, Brian F. Cumming, Andrew M. Paterson,
Kathleen Rühland, Alexander P. Wolfe, and John P. Smol. 2009. ``Do
Spectrally Inferred Determinations of Chlorophyll a Reflect Trends in
Lake Trophic Status?'' \emph{Journal of Paleolimnology} 43 (2). Springer
Nature: 205--17.
doi:\href{https://doi.org/10.1007/s10933-009-9325-8}{10.1007/s10933-009-9325-8}.


\end{document}
